\section{学途伊始}
最开始学习MATLAB是从脚本开始的,写一段然后就选中代码按F9这种.后来逐渐需
要把一些通用或常用的脚本写成函数,需要的时候一条语句就可以实现.这大概也
就是整体上我的学途历程了.\par
再往后,就开始考虑效率,接触一些方便好用的功能,考虑算法设计和
MATLAB编程特性的结合,如何去调试程序等等.

\subsection{窗口}
MATLAB有几个基本的窗口: Command Window用于输出结果以及执行命令之用; Editor
则作为编辑代码之用;在Workspace可以看到当前保存的变量; Command History保
存了执行过的命令的历史记录; Current Folder是当前目录,方便了文件和资源管理.

\subsection{脚本}
在MATLAB里面Ctrl+N新建一个后缀
为m的文件,也就是通常说的M-file.写入要实现的功能,选中要执行的代码然后按F9就可
以在Command窗口查看输出结果,或者在Workspace中查看保存的变量.比如,下面给出的例子就包含了注释、一个输出语句以及一
个矩阵的创建.

\vspace{-0.8cm}
\begin{lstlisting}[caption=第一个脚本]
  disp('Hello MATLAB') % 我是注释,前面一条语句是输出
  A = [1 2; 3 4];      % 创建一个矩阵,若句尾不加分号执行时会直接输出
\end{lstlisting}

\subsection{函数}
把常用到的功能或者具有一定功能的单独拿出来,写成一个函数,在需要的时候直
接调用.这一方便避免的代码的重复书写,又方便了程序的调试.函数不能在当前
文件执行,需要通过调用的形式来执行.比如下面的这个函数保存为文件之后,可
以直接在Command窗口调用.

\vspace{-0.8cm}
\begin{lstlisting}[caption = 第一个函数]
  function sm = twonosum(no1,no2)
  % 两个数之和
  sm = no1 + no2;
  end
\end{lstlisting}

\begindot
  \item 保存时需要和函数名一致,上面的例子应保存为twonosum.m;
  \item Command窗口或脚本调用格式为 \mcode{var1 = twonosum(var2,
      var3)};
  \item 文件twonosum.m所在路径需要添加,这在文中的路径设置中提到;
\myenddot

\notation{最后的 \mcode{end} 并非是必须的,可以删除.}