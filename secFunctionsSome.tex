\section{一些函数}

\paragraph{help} 查看帮助文档.比如:\mcode{help size}.
\paragraph{prod} 向量元素连乘,或矩阵元素的列元素连乘
\paragraph{eval}
\paragraph{pause} 暂停,暂停5秒,即\mcode{pause(5)}.
\paragraph{nnz} 零元素的个数,由此可以获得其他元素的个数(通过减法)
\paragraph{xlsread xlswrite}
\paragraph{clc} 清屏,清除Command窗口显示内容.
\paragraph{clear} 清除变量,清楚工作空间存储的变量.其后加变量名可以只清
除指定变量,比如:\mcode{clear vari}.
\paragraph{pack} 整理内存.
\paragraph{peaks} 画出一个特殊的图形来...
\paragraph{logo} MATLAB的标志.
\paragraph{version} 获得MATLAB的版本号.
\paragraph{ver} 获得MATLAB各组件的版本号.
\paragraph{load} 从硬盘加载数据.
\paragraph{save} 保存工作空间变量.
\paragraph{diary} 保存Command屏幕输出,由于屏幕输出延时,可能并没有保
存数据,因此有两个方案来应对;其一,加入pause;其二,判断是否已保存数
据,直到有数据才执行diary off.
\paragraph{exit} 退出MATLAB.
\paragraph{which} 查看函数的文件路径.比如:\mcode{which fmincon}.
\paragraph{whos} 查看当前工作空间变量的属性.查看某个变量的属性这添加变
量名作为参数,比如:\mcode{whos var} 指查看 \mcode{var} 变量的属性.