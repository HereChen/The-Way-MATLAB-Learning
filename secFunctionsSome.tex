\section{一些函数}
对于要介绍的函数或命令,在每个说明后面紧接着给出示例,或者是使用是说明.

\begin{itemize}
\item{eval} -- 将字符串作为MATLAB语句执行. \mcode{eval('[1 2]*3')}
\item{help} -- 查看帮助文档. 查看 \mcode{size} 函数帮助, \mcode{help size}
\item{prod} -- 向量元素的连乘. 矩阵同行元素相乘, \mcode{prod([1 2;3 4],2)}
\item{pause} -- 暂停.暂停5秒, \mcode{pause(5)}
\item{nnz} -- 矩阵非零元素的个数. \mcode{nnz([0 0 0 1 2])
}
\item{nonzeros} -- 矩阵非零元素. \mcode{nonzeros([0 0 0 1 2])}
\item{xlsread} -- Excel文件读取.调用格式 \mcode{[NUM,TXT,RAW]=xlsread(FILE,SHEET,RANGE)}
\item{xlswrite} -- Excel文件写入.如果不加SHEET参数则存储到默认的表格,若是自定义则会在Excel中添加新的表格.调用格式 \mcode{xlswrite(FILE,ARRAY,SHEET)}
\item{clc} -- 清屏.清除Command窗口显示内容. \mcode{clc}
\item{clear} -- 清除变量,清楚工作空间存储的变量.其后加变量名可以只清
除指定变量,清除全部变量直接使用即可,清除某个变量vari, \mcode{clear vari}
\item{save} -- 保存工作空间变量.保存全部变量, \mcode{test.mat};保存某几个变量,\\\mcode{p = 1; q = [1 2]; save('data.mat', 'p', 'q');}
\item{load} -- 从硬盘加载数据. 用法和 \mcode{save} 类似.
\item{diary} -- 保存Command屏幕输出,由于屏幕输出延时,可能并没有保
存数据,因此有两个方案来应对;其一,加入pause;其二,判断是否已保存数
据,直到有数据才执行 \mcode{diary off}
\item{which} -- 查看函数的文件路径. 查看函数 \mcode{fmincon} 的位置, \mcode{which fmincon}
\item{whos} -- 查看当前工作空间变量的属性.直接使用为查看全部变量属性,查看 \mcode{var} 变量的属性, \mcode{whos var}
\item{saveas} -- 保存图像(Figure或仿真框图).保存当前figure为jpg图像, \\\mcode{saveas(gcf, 'output', 'jpg')}
\item{textscan} -- 格式化读取文件或字符串.
\item{regexp} -- 用正则表达式匹配内容.
\end{itemize}