\section{高效编程}

\subsection{分析代码性能}

通过Profiler这个工具来查看执行程序中占用时间比例,并作有针对性的优化这样可以。

\vspace{-0.8cm}
\lstinputlisting[language=matlab, caption={用Profiler分析代码性能}]{secEfficientCoding/open-profile.m}

分析结果包括每部分的占用时间比例,没个函数或语句的执行时间,点击链接可以进入到下一层,以此来逐层查看。

\notation{也可以通过在Profiler中的Run this code输入要分析的文件名或代码,点击Start Profiling按钮开始分析。打开Profiler的方式除了执行 \mcode{profile viewer},还可以通过在HOME下的Run and Time按钮来打开。}

% \subsection{数据类型}
%  double,cell





\subsection{预分配内存}

为矩阵(或向量)预分配内存,并尽量避免改变矩阵的大小。为什么预分配内存呢?实际上,申请新的内存本身就是一种耗费。每当申请需要新的内存存储变量时,MATLAB 都会先查找是否有足够并且逻辑上是连续的内存空间来存储(若没有则会内存溢出,无法继续执行程序)。 对于矩阵,每次对它添加元素时,其所占内存就在改变,这种改变在地址上的本质是,它不是在原有的基础上添加,而是找到一块合适的地址之后,存储到新的空间上,并删除原来空间上的数据。

\vspace{-0.8cm}
\lstinputlisting[language=matlab, caption={是否预分配内存效率对比}]{secEfficientCoding/memory-pre-allocation.m}

\vspace{-0.8cm}
\lstinputlisting[language=matlab, nolol]{secEfficientCoding/memory-pre-allocation-result.m}





\subsection{临时变量}

这里所谓临时变量,是指那些并不是关键的要存储的变量,而只是临时使用。尽量少使用临时变量,或者使用少量的临时变量,毕竟申请内存是一件耗费的事情。





\subsection{MATLAB 建议}

根据 MATLAB 建议编写程序(有时候编程会出现黄色下划线)。但不更改也并不会影响程序的执行。这些建议包括一些函数使用的提醒,比如将来会移除这个函数,也包括联系上下文实现某一功能时的参数建议,还包括变量的使用建议。





\subsection{矩阵存储方式}

矩阵的存储方式以列优先存储,即我们在计算时尽量采用列优先的方式。

\vspace{-0.8cm}
\lstinputlisting[language=matlab, caption={矩阵不同存取方式效率对比}]{secEfficientCoding/matrix-col-row.m}

\vspace{-0.8cm}
\lstinputlisting[language=matlab, nolol]{secEfficientCoding/matrix-col-row-result.m}





\subsection{函数类型}

不同函数类型的执行效率并一样,在此仅以符号函数和匿名函数作为对比对象。

\vspace{-0.8cm}
\lstinputlisting[language=matlab, caption={不同函数类型效率对比}]{secEfficientCoding/functions.m}

\vspace{-0.8cm}
\lstinputlisting[language=matlab, nolol]{secEfficientCoding/functions-result.m}





\subsection{大数据处理}

直接的描述就是提前给大数据分配内存。过多小数据分配内存后形成的内存零碎化,这将可能在后面为大数据寻找合适的连续内存地址消耗更多时间。





\subsection{并行计算}

这里所指的并行计算是针对一台多核的电脑。所谓并行计算,是指让多个工作同时进行,从而节省计算时间。这里以 \mcode{parfor} 作为解释对象,要用到的是 \mcode{matlabpool}。\par

\mcode{parfor} 即 parallel for (并行的 \mcode{for} 语句),可以看作是把 \mcode{for} 循环执行的内容改造成了同时执行的内容。 \mcode{matlabpool} 用于打开多核,默认是并没有打开的。\par

实现并行计算包括两个操作,一是打开多核(或者worker),二是要保证循环计算之间并不存在关联性。

\vspace{-0.8cm}
\lstinputlisting[language=matlab, caption={并行计算parfor与for效率对比}]{secEfficientCoding/parfor-for.m}

\vspace{-0.8cm}
\lstinputlisting[language=matlab, nolol]{secEfficientCoding/parfor-for-result.m}

尽管通常不是用循环来实现的内容,通过转化为不相关的循环计算,就可以用并行计算来实现。

\notation{在MATLAB右下角可以看到打开worker的数量。}





% \subsection{混合编程}
%  预留.....




% \subsection{C/C++ 代码生成}
%  预留.....





\subsection{其他}

\begin{itemize}
    \item \textbf{线性方程组求解} \href{https://zhuanlan.zhihu.com/p/30958676}{MATLAB R2017b新函数介绍之decomposition(加速X$\backslash$y)}。沒有新版的MATLAB,无法进行测试. 基本思路是通过分解矩阵加速线性方程组的求解,而\mcode{decomposition}可以根据矩阵特点选择合适的分解方法。方程组求解效率可参照数值分析中的消元法和矩阵分解法(比如LU)。
\end{itemize}