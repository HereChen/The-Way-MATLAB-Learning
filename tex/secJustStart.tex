\section{学途伊始}
\setcounter{page}{1}

最开始学习MATLAB是从脚本开始的,写一段然后就选中代码按F9这种。后来逐渐需要把一些通用或常用的脚本写成函数,需要的时候一条语句就可以实现。这大概也就是整体上我的学途历程了。\par

再往后,就开始考虑效率,接触一些方便好用的功能,考虑算法设计和MATLAB编程特性的结合,如何去调试程序等等。





\subsection{窗口}

MATLAB有几个基本的窗口:Command Window用于输出结果以及执行命令之用;Editor则作为编辑代码之用;在Workspace可以看到当前保存的变量;Command History保存了执行过的命令的历史记录;Current Folder是当前目录,方便了文件和资源管理。





\subsection{脚本}

在MATLAB里面Ctrl+N新建一个后缀为m的文件,也就是通常说的M-file。写入要实现的功能,选中要执行的代码然后按F9就可以在Command窗口查看输出结果,或者在Workspace中查看保存的变量。比如,下面给出的例子就包含了注释、一个输出语句以及一个矩阵的创建。

\vspace{-0.8cm}
\lstinputlisting[language=matlab, caption={第一个脚本}]{secJustStart/first-script.m}






\subsection{函数}

把常用到的功能或者具有一定功能的单独拿出来,写成一个函数,在需要的时候直接调用。这一方便避免的代码的重复书写,又方便了程序的调试。函数不能在当前文件执行,需要通过调用的形式来执行。比如下面的这个函数保存为文件之后,可以直接在Command窗口调用。

\vspace{-0.8cm}
\lstinputlisting[language=matlab, caption={第一个函数}]{secJustStart/first-function.m}

\begindot
  \item 保存时需要和函数名一致,上面的例子应保存为 twonosum.m;
  \item Command 窗口或脚本调用格式为 \mcode{var1 = twonosum(var2, var3)};
  \item 文件 twonosum.m 所在路径需要添加,这在文中的路径设置中提到。
\myenddot

\notation{最后的 \mcode{end} 并非是必须的,可以删除。}
