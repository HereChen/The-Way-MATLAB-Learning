\section{高效编程}



\subsection{分析代码性能}
 通过Profiler这个工具来查看执行程序中占用时间比例,并作有针对性的优化这样可以.

  \vspace{-0.8cm}
  \begin{lstlisting}[caption = 用Profiler分析代码性能] 
  profile on                        % 打开profile
  plot(magic(35))                   % 要执行的代码
  profile viewer                    % 查看分析结果
  \end{lstlisting}

分析结果包括每部分的占用时间比例,没个函数或语句的执行时间,点击链接可以进入到下一层,以此来逐层查看.

\notation{也可以通过在Profiler中的Run this code输入要分析的文件名或代码,点击Start Profiling按钮开始分析.打开Profiler的方式除了执行 \mcode{profile viewer},还可以通过在HOME下的Run and Time按钮来打开.}

% \subsection{数据类型}
%  double,cell



\subsection{预分配内存}
 为矩阵(或向量)预分配内存,并尽量避免改变矩阵的大小.为什么预分配内存呢?实际上,申请新的内存本身就是一种耗费.每当申请需要新的内存存储变量时,MATLAB 都会先查找是否有足够并且逻辑上是连续的内存空间来存储(若没有则会内存溢出,无法继续执行程序). 对于矩阵,每次对它添加元素时,其所占内存就在改变,这种改变在地址上的本质是,它不是在原有的基础上添加,而是找到一块合适的地址之后,存储到新的空间上,并删除原来空间上的数据.

\vspace{-0.8cm}
\begin{lstlisting}[caption = 是否预分配内存效率对比]
n = 1e+7;

%% 未初始化赋值
tic
for i=1:n
    a(i) = i;
end
toc

%% 初始化后赋值
tic
b = zeros(1,n,'double');
for i=1:n
    b(i) = i;
end
toc
\end{lstlisting}

\vspace{-0.8cm}
\begin{lstlisting}
Elapsed time is 4.193847 seconds.
Elapsed time is 0.177962 seconds.
\end{lstlisting}



\subsection{临时变量}
 这里所谓临时变量,是指那些并不是关键的要存储的变量,而只是临时使用.尽量少使用临时变量,或者使用少量的临时变量,毕竟申请内存是一件耗费的事情.



\subsection{MATLAB 建议}
 根据 MATLAB 建议编写程序(有时候编程会出现黄色下划线).但不更改也并不会影响程序的执行.这些建议包括一些函数使用的提醒,比如将来会移除这个函数,也包括联系上下文实现某一功能时的参数建议,还包括变量的使用建议.



\subsection{矩阵存储方式}
 矩阵的存储方式以列优先存储,即我们在计算时尽量采用列优先的方式.

\vspace{-0.8cm}
\begin{lstlisting}[caption = 矩阵不同存取方式效率对比]
n = 10000;
A = rand(n , n);
B = zeros(n , n);

%% 以行存取
tic
for i = 1:n
    B(i , :) = A(i , :);        % 以行方式赋值 
end
toc

%% 以列存取
tic
for i = 1:n
    B(: , i) = A(: , i);        % 以列方式赋值
end
toc
\end{lstlisting}

  \vspace{-0.8cm}
  \begin{lstlisting}
    Elapsed time is 24.323625 seconds.
    Elapsed time is 0.621799 seconds.
  \end{lstlisting}

\subsection{函数类型}
 不同函数类型的执行效率并一样,在此仅以符号函数和匿名函数作为对比对象.

  \vspace{-0.8cm}
  \begin{lstlisting}[caption = 不同函数类型效率对比]
    %% 符号函数
    syms fun1(x);
    fun1(x) = x^3 + x^2 + x +1;
    tic
    fun1(5);
    toc

    %% 匿名函数
    fun2 = @(x)x^3 + x^2 + x +1;
    tic
    fun2(5);
    toc
  \end{lstlisting}

  \vspace{-0.8cm}
  \begin{lstlisting}
    Elapsed time is 0.006139 seconds.
    Elapsed time is 0.000074 seconds.
  \end{lstlisting}



\subsection{大数据处理}
 直接的描述就是提前给大数据分配内存. 过多小数据分配内存后形成的内存零碎化,这将可能在后面为大数据寻找合适的连续内存地址消耗更多时间.



\subsection{并行计算}
 这里所指的并行计算是针对一台多核的电脑.
 所谓并行计算,是指让多个工作同时进行,从而节省计算时间.这里以 \mcode{parfor} 作为解释对象,要用到的是 \mcode{matlabpool}.\par
 \mcode{parfor} 即 parallel for (并行的 \mcode{for} 语句),可以看作是把 \mcode{for} 循环执行的内容改造成了同时执行的内容. \mcode{matlabpool} 用于打开多核,默认是并没有打开的.\par
 实现并行计算包括两个操作,一是打开多核(或者worker),二是要保证循环计算之间并不存在关联性.

  \vspace{-0.8cm}
  \begin{lstlisting}[caption = 并行计算parfor与for效率对比]
    matlabpool open 2               % 打开两个worker
    M = 200; a = zeros(M,1);        % 200 次计算

    %% for
    tic  
    for i = 1:M        
        a(i) = max(eig(rand(M)));   % 取随机矩阵最大特征值
    end 
    toc
    
    %% parfor
    tic  
    parfor i = 1:M 
        a(i) = max(eig(rand(M))); 
    end 
    toc
    matlabpool close                % 关闭所有worker
  \end{lstlisting}

  \vspace{-0.8cm}
  \begin{lstlisting}
    Elapsed time is 11.092022 seconds.
    Elapsed time is 6.325050 seconds.
  \end{lstlisting}

尽管通常不是用循环来实现的内容,通过转化为不相关的循环计算,就可以用并行计算来实现.

\notation{在MATLAB右下角可以看到打开worker的数量.}



% \subsection{混合编程}
%  预留.....